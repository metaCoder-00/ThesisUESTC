
\documentclass[master]{thesis-uestc}

\title{数字相控阵单脉冲测向方法研究}{The Research on Monopulse estimation with 
Digital Phased Array}

\author{邓宇昊}{Yuhao Deng}
\advisor{谢菊兰\chinesespace 副教授}{Dr. Julan Xie}
\school{信息与通信工程学院}{School of Information and Communication Engineering}
\major{信号与信息处理}{Signal and Information Processing}
\studentnumber{201821011229}

\begin{document}

\makecover

\begin{chineseabstract}
这是摘要

……

\chinesekeyword{关,键,词}
\end{chineseabstract}

\begin{englishabstract}
This is abstract.

\englishkeyword{key, words}
\end{englishabstract}

\thesistableofcontents

\thesischapterexordium

\section{研究工作的背景与意义}
计算电磁学方法\citing{wang1999sanwei, liuxf2006, zhu1973wulixue, chen2001hao, gu2012lao, feng997he}
从时、频域角度划分可以分为频域方法与时域方法两大类。频域方法的研究开展较早,
目前应用广泛的包括:矩量法(MOM)\citing{xiao2012yi,zhong1994zhong}及其快速算法多层快速多极子(MLFMA)
\citing{clerc2010discrete}方法、有限元(FEM)\citing{wang1999sanwei,zhu1973wulixue}方法、
自适应积分(AIM)\citing{gu2012lao}方法等,
这些方法是目前计算电磁学商用软件\footnote{脚注序号“\ding{172},……,\ding{180}”的字体是“正文”,不是“上标”,
序号与脚注内容文字之间空1个半角字符,脚注的段落格式为:单倍行距,段前空0磅,段后空0磅,悬挂缩进1.5字符;
中文用宋体,字号为小五号,英文和数字用Times New Roman字体,字号为9磅;
中英文混排时,所有标点符号(例如逗号“,”、括号“()”等)一律使用中文输入状态下的标点符号,
但小数点采用英文状态下的样式“.”。}(例如:FEKO、Ansys 等)的核心算法。
由文献\cite{feng997he,clerc2010discrete,xiao2012yi}可知

\section{单脉冲方法的国内外研究历史与现状}
时域积分方程方法的研究始于上世纪60 年代,

\section{本文的主要贡献与创新}
本论文以时域积分方程时间步进算法的数值实现技术、后时稳定性问题以及两层平面波加速算法为重点研究内容,主要创新点与贡献如下:

\section{本论文的结构安排}
本文的章节结构安排如下:

\chapter{相控阵单脉冲测向基本理论}
时域积分方程(TDIE)方法作为分析瞬态电磁波动现象最主要的数值算法之一,常用于求解均匀散射体和表面散射体的瞬态电磁散射问题。

\section{相控阵接收信号模型}
这是模型

\section{波束形成技术}
利用数值算法求解时域积分方程,首先需要选取适当的空间基函数与时间基函数对待求感应电流进行离散。

\section{传统单脉冲方法}
方法
\subsection{半阵测向}

\subsection{加权测向}

\subsection{和差比幅}

\section{本章小结}
本章结。

\chapter{相控阵的非自适应单脉冲方法}

\section{1}
时域积分方程时间步进算法的阻抗元素直接影响算法的后时稳定性,因此阻抗元素的计算是算法的关键之一,采用精度高效的方法计算时域阻抗元素是时域积分方程时间步进算法研究的重点之一。

\section{2}
由于时域混合场积分方程是时域电场积分方程与时域磁场积分方程的线性组合,因此时域混合场积分方程时间步进算法的阻抗矩阵特征与时域电场积分方程时间步进算法的阻抗矩阵特征相同。时域阻抗元素的存储技术也是关键技术之一,采用合适的阻抗元素存储方式可以提高并行算法的计算效率。

\section{3}
由于时域混合场积分方程是时域电场积分方程与时域磁场积分方程的线性组合,因此时域混合场积分方程时间步进算法的阻抗矩阵特征与时域电场积分方程时间步进算法的阻抗矩阵特征相同。

\begin{algorithm}[H]
    \KwData{this text}
    \KwResult{how to write algorithm with \LaTeX2e}
    initialization\;
    \While{not at end of this document}{
        read current\;
        \eIf{understand}{
            go to next section\;
            current section becomes this one\;
        }{
            go back to the beginning of current section\;
        }
    }
    \caption{How to wirte an algorithm.}
\end{algorithm}

由于时域混合场积分方程是时域电场积分方程与时域磁场积分方程的线性组合,
因此时域混合场积分方程时间步进算法的阻抗矩阵特征与时域电场积分方程时间步进算法的阻抗矩阵特征相同。
\section{本章小结}
本章首先研究了时域积分方程时间步进算法的阻抗元素精确计算技术,分别采用DUFFY变换法与卷积积分精度计算法计算时域阻抗元素,通过算例验证了计算方法的高精度。

\chapter{相控阵的自适应单脉冲方法}
\section{最大似然方法}
时域积分方程时间步进算法的阻抗元素直接影响算法的后时稳定性,
因此阻抗元素的计算是算法的关键之一,
采用精度高效的方法计算时域阻抗元素是时域积分方程时间步进算法研究的重点之一。

\section{MVAM方法}
时域阻抗元素的存储技术也是时间步进算法并行化的关键技术之一,
采用合适的阻抗元素存储方式可以很大的提高并行时间步进算法的计算效率。

\section{线性约束方法}
由于时域混合场积分方程是时域电场积分方程与时域磁场积分方程的线性组合,
因此时域混合场积分方程时间步进算法的阻抗矩阵特征与时域电场积分方程时间步进算法的阻抗矩阵特征相同。

\section{SVD-线性约束}
如表\ref{tablea}所示给出了时间步长分别取0.4ns、0.5ns、0.6ns 时的三种存储
方式的存储量大小。

\section{本章小结}
结
\chapter{极化相控阵的单脉冲测向方法}

\section{极化相控阵接收信号模型}
本文以时域积分方程方法为研究背景,主要对求解时域积分方程的时间步进算法以及两层平面波快速算法进行了研究。
\subsection{各阵元摆放角度一致的接收信号模型}

\subsection{各阵元摆放角度不同的接收信号模型}

\section{各阵元摆放角度一致的极化相控阵单脉冲测向}
时域积分方程方法的研究近几年发展迅速,在本文研究工作的基础上,仍有以下方向值得进一步研究:

\subsection{原理}

\subsection{性能分析}

\subsection{仿真}

\section{各阵元摆放角度不同的极化相控阵单脉冲测向}

\subsection{原理}

\subsection{仿真}

\section{本章小结}

\chapter{全文总结与展望}

\section{全文总结}

\section{后续工作展望}

\thesisacknowledgement
在攻读博士学位期间,首先衷心感谢我的导师XXX教授

\thesisappendix

\chapter{中心极限定理的证明}

\section{高斯分布和伯努利实验}


% Uncomment to list all the entries of the database.
% \nocite{*}

\thesisbibliography{reference}


% Uncomment following codes to load bibliography database with native
% \bibliography command.

% \nocite{*}
% \bibliographystyle{thesis-uestc}
% \bibliography{reference}

% \thesisaccomplish{publications}

% \thesistranslationoriginal
% \section{The OFDM Model of Multiple Carrier Waves}

% \thesistranslationchinese
% \section{基于多载波索引键控的正交频分多路复用系统模型}

\end{document}
